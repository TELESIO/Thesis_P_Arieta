\documentclass[tesi]{subfiles}

\IfEq{\jobname}{\detokenize{tesi}}{}{%
	\externaldocument{Chapter1}    
    \externaldocument{Chapter4}
}
\begin{document}

\chapter{System Development}
\label{ch:System Development}
This chapter analyses how the system has been developed, mainly focusing on:

\begin{itemize}
\item How data are collect.
\item Processing which they are subject to, and how the indexes are determined.
\item Data storing.
\item Viewing on interactive map.
\end{itemize}


\clearpage
\noindent The figure below shows the structure of the system.
\begin{figure}[H]
\centering
\includegraphics[scale=0.6]{WorkFlow}
\caption{System Structure}

\end{figure}\label{fig:System Structure}

\section{Data Collect}\label{sc:Data Collect}
Initially, the smartphone was fixed up at the windshield of the car by an arm support, forming a $90°$ angle between the phone and the vehicle axes.\\
However, during travel, the support was subject to vibrations that caused additional noise in the data, and could also be subject to movements due to the nature of the support itself and the road surface conditions, thus changing the integrity of the data.\\
The smartphone was then mounted horizontally on the car dashboard, forming a proximal $0°$ angle with the same, using a non-slip mat, where the perceived vibrations are proportional at the vibrations sensed by the car, the movements of the smartphones are almost null.\\
The AndroSensors application available on the PlayStore was used to make the measurements. That allows us to perform contemporary measurements of various sensors.\\
For our purpose were monitoring:

\begin{itemize}
\item Accelerometer.
\item Linear Accelerometer.
\item GPS.
\item Orientation.
\item Date
\item Time
\end{itemize}


\noindent The data are updated at the highest frequency \textit{"Very Fast"}, while the sampling frequency can be chosen within a range that goes:


\begin{center}
$fs_{min} \quad <= \thinspace f_{s} \thinspace <= \thinspace fs_{max}$\\
$200 \thinspace Hz \quad <= \thinspace f_{s} \thinspace <= \thinspace 1 \thinspace Hz$\\
$0,005 \thinspace s <= \thinspace f_{s} \thinspace <= \thinspace 1 \thinspace s $
\end{center}


\noindent Choosing a very low sampling frequency (for example, $fs_{min}$, sensor values may be inconsistent at writing time due to their fast sampling frequency, however, the higher is the sample rate, higher is the accuracy and quality of the information.\\
Conversely, choosing a very high sampling frequency, (for example, $fs_{max}$), we will have very disconnected data, in fact, if you are supposed to travel at a speed of $130 \thinspace \si{\km\per\hour}$, you will have a data every $36,11  \si{\meter}$ of road, which is not favorable to the monitoring of road surface conditions.\\
Because of these reasons, was chosen a sampling frequency of:
\begin{center}
$fs = 100 \thinspace Hz; \quad fs = 0,01s; \quad fs = 10ms$
\end{center}

\noindent Which is stable, and for every second of recording we own $100$ sample of data for the previously listened sensors. The GPS still have the same number of data, but changing every second as the GPS sensor updates at a frequency of $1 \thinspace Hz$ (in some cases this frequency may be lower by reaching a maximum of $20 \thinspace Hz$).\\
Once the GPS signal has been established, it is possible to start recording the data.
 
 
 
 \section{Data Processing}\label{sc:Data Processing}
 Regarding data processing, 3 indexes will be extrapolated:
 
 \begin{itemize}
 \item Simple Accelerations Points
 \item Critical Points
 \item IRI
 \end{itemize}


\noindent Once the measurements are complete, the data will be saved inside a $.xls$ file

\noindent These will be processed with MATLAB (Matrix Laboratory) a multi-paradigm numerical computing environment, which makes numerical computing more easier and computationally faster than other programming paradigms.

\noindent Initially, the $.xls$ file will be read, and each column of it depending on the data nature (numeric or string) will be stored in a vector, which can be found in the MATLAB workspace.

\noindent Once the raw data has been read, it is possible starting calculating the indices.


\subsection{Simple Acceleration Points}\label{ssc:Simple Accelerations Points}
This index helps principally to show how the acceleration component changes depending on different points on the road surface.\\It notes that an appropriate methodology is needed to calculate the surface conditions because the representation of the acceleration signal alone would not be satisfactory to provide all the conditions of a given road segment.\\
This index will be calculated on a segment of a predetermined length ($distance_{segment}$) in which an $\chi$ value will be associated and calculated as follows:

\begin{center}
 $\chi \thinspace = \left( \sum \limits_{i=1}^{N} \thinspace a_{i} \right) \thinspace \dfrac{1}{N}$
\end{center}\label{eq:sap}


\noindent Where: $N$, is the number of GPS points necessary to reach a lenght of $distance_{segment}$, and $a_{i}$ is the processed acceleration value associated to $i$.


\noindent For example, if considering an urban road of more or less good conditions, the travel speed on it can not be elevated, so for each segment considered, we will have many values associated with it to handle during the processing, belong to several seconds of registration.\\
\noindent This means that at the time of the final calculation of the value, it will be averaged with many points. The end result will definitely have a small value. This value would indicate that the condition of the road surface it is almost perfect, although it is not really so, since considering urban roads, they may have various abnormalities.\\\\
\noindent Conversely, if taking into consideration a highway, the travel speed will be much higher so for each segment taken into considerations, the values to be handled are lower respect to a urban road. The final result will have a higher value, which suggests that road surface conditions are not good, although the surface of a highway is very comfortable and almost flat.\\\\
\noindent This value will be calculated by performing a few processing operations on it.\\\\


\noindent Considering the following signal:
\begin{figure}[H]
\centering
\includegraphics[scale=0.16]{SPARaw}
\caption{Raw Accelerometer Signal}
\end{figure}

\noindent The steps are as follows:

\begin{description}
\item[1. Accelerometer Reorentation:] First of all it is applied the procedure of Accelerometer reorientation explained in Chapter\ref{ch:Data Analysis} (section:\ref{sc:Accelerometer Reorientation}, on page: \pageref{sc:Accelerometer Reorientation}).
\item[2. GPS points division:] Next, the GPS points are subdivided according to the methodology explained in Chapter\ref{ch:Data Analysis} (section:\ref{sc:GPS points division}, on page: \pageref{sc:GPS points division}).
\item[3. Remove Engine Vibrations Filter:] This filter is the first operation that is performed on the data, in which the noise components generated by the engine will be smoothed, according to the application seen in the Chapter\ref{ch:Data Analysis} (section:\ref{sc:Data Filtering}, on page:\pageref{sssc:Remove Engine Vibrations Filter})\\
A figure below show how the filter is applied on the Signal
\begin{figure}[H]
\centering
\includegraphics[scale=0.16]{SPANoEngine}
\caption{Signal after application of the filter}
\end{figure}
\item[4. Zero Velocity Filter:] After applying the first filter, this is also applied, as it is explained in the Chapter\ref{ch:Data Analysis} (section:\ref{sc:Data Filtering}, on page:\pageref{sssc:Zero Velocity Filter}).\\
A figure below show how the filter is applied on the Signal
\begin{figure}[H]
\centering
\includegraphics[scale=0.16]{SPANoVelocity}
\caption{Signal after application of the filter}
\end{figure}
\item[5. Calculation of the final index:] Ultimately, the final index is calculated. 
This value it is associated with a certain portion of the road ($distance_{segment}$). Following the application of the two previous filters, the Haversine formula will be used (as it is explanained on page: \ref{ssc:Haversine Formula}) to calculate the cumulative GPS distance of points until the $distance_{segment}$ is reached. The new GPS point will be identified by the set of points needed to compose the segment. For the $\chi$ \thinspace \ref{eq:sap} value, it will be calculated by all the acceleration values that are simultaneously read together the GPS points.\\
A figure below show the result.
\begin{figure}[H]
\centering
\includegraphics[scale=0.16]{SPAFinal}
\caption{Final Result}
\end{figure}\label{fig:Simple Accelerations Points Final Result}
\end{description}


\subsection{Critical Points}\label{ssc:Critical Points}
This index is very useful as it allows us to locate the most damaged points on the road surface.\\
\noindent Allowing us to identify holes, bumps, and all those types of anomalies that passing over one of them could damage the vehicle and cause an inadequate feeling level of comfort at the driver \ref{ch:Introduction}.\\For this, be informed of the localisation of these anomalies,  allows the driver during navigation, to avoid them or reduce speed in their proximity.\\
\noindent This index is processed and calculated only by the vertical acceleration signal because the presence of high peaks corresponds to high-energy events generated by road-vehicle vibration and can, therefore, be associated like "anomalies" on the surface.\\\\
\noindent Each of these points will be properly geolocated on the Earth's surface, as for the other indexes, also in this case, at each acceleration signal, GPS coordinates will be associated.	\\\\
\noindent Following the processing of the signal, which will be properly cleaned, in order to identify the critical points, the final value will be calculated as follows: 
\begin{center}
${\large \kappa_{i} \thinspace = 0 \qquad	if \quad	threshold_{min} \thinspace <= \thinspace a_{i} \thinspace <= \thinspace threshold_{max}}$\\

${\large \thinspace \kappa_{i} \thinspace = a_{i} \qquad if \quad a_{i} \thinspace > \thinspace threshold_{max} \quad or \quad a_{i} \thinspace < \thinspace threshold_{min}}$
\end{center}

\noindent Where $\kappa_{i}$ is the index results, $a_{i}$ is the acceleration signal. Following several inspections and controls on the signal, it was possible to identify two thresholds $\left(threshold_{min},\quad threshold_{max}\right)$. 
If the signal falls within this range then it does not correspond to an anomaly, vice-versa if it is outside it, represent an anomaly.\\\\

\noindent Considering the following signal:

\begin{figure}[H]
\centering
\includegraphics[scale=0.16]{CPRaw}
\caption{Original Signal}
\end{figure}


\noindent The steps for its calculation are:
\begin{description}
\item[1. Accelerometer Reorentation:] First of all it is applied the procedure of Accelerometer reorientation explained in Chapter\ref{ch:Data Analysis} (section:\ref{sc:Accelerometer Reorientation}, on page: \pageref{sc:Accelerometer Reorientation}).
\item[2. GPS points division:] Next, the GPS points are subdivided according to the methodology explained in Chapter\ref{ch:Data Analysis} (section:\ref{sc:GPS points division}, on page: \pageref{sc:GPS points division}).
\item[3. Remove Engine Vibrations Filter:] This filter is the first operation that is performed on the data, in which the noise components generated by the engine will be smoothed, according to the application seen in the Chapter\ref{ch:Data Analysis} (section:\ref{sc:Data Filtering}, on page:\pageref{sssc:Remove Engine Vibrations Filter}).\\
The result of the application of the filter on the signal shown above, is:
 \begin{figure}[H]
\centering
\includegraphics[scale=0.16]{CPNoEngine}
\caption{Signal Without Engine Vibrations}
\end{figure}
\item[4. Zero Velocity Filter:] After applying the first filter, this is also applied, as it is explained in the Chapter\ref{ch:Data Analysis} (section:\ref{sc:Data Filtering}, on page:\pageref{sssc:Zero Velocity Filter}).\\
The result of the application of the filter on the signal shown above, is:
 \begin{figure}[H]
\centering
\includegraphics[scale=0.16]{CPNoVelocity}
\caption{Null Velocity Removed}
\end{figure}
\item[5. Determining Critical Points:] In this step, the critical points are identified, according to the formula explained above. The result is shown in the figure.
The result of the application of this filter on the signal shown above, is:
 \begin{figure}[H]
\centering
\includegraphics[scale=0.16]{CPUnderTH}
\caption{Results Final}
\end{figure}\label{fig:Critical Points Final Result}
\item[6. Grouping the points too close each other:]  Defined the critical points, there is a final step to be carried out.\\Following the GPS points division(\ref{sc:GPS points division}), it may happen in the previous step, that consecutive points are selected, or in any case too close each other, which refers to the same anomaly (in fact, the signal is similar for a given small time due the high recording frequency, so the same anomaly would have more associated signals).\\It is necessary to incorporate these signals into a single point, for this purpose, the Haversine formula(\ref{ssc:Haversine Formula}) was used.\\When a set of points is identified, they will be grouped within a certain distance and enclosed within a single GPS point.\\Similarly to the final $\kappa$ value, it will be identified as the average of the points belonging to this set, because they refer to the same anomaly and will have similar values.
\end{description}




\end{document}