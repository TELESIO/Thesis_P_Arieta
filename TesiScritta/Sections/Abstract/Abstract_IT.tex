\documentclass[tesi]{subfiles}
\begin{document}
\begin{flushright}
\section*{Abstract}\label{Abstract_IT}
\vspace{10mm}
\end{flushright}
In questo lavoro, verrà esaminato come migliorare la sicurezza stradale, tramite la raccolta e la distribuzione di informazioni relative alle condizioni della superfice, attraverso l’utilizzo di sensori a basso costo, al fine di fornire informazioni utili, sia agli utenti della strada che alle istituzioni pubbliche per la manutenzione della rete stradale. È stato sviluppato un sistema per la misurazione di alcuni indici, quali l’International Roughness Index (IRI) e la localizzazione dei punti maggiormente critici. La descrizione della qualità della superfice stradale è stata condotta mediante l’analisi di un profilo longitudinale. In particolare, a tal fine, è stato utilizzato uno smartphone, correttamente posizionato all’interno dell’abitacolo, dal quale vengono registrati e processati i dati prodotti da alcuni sensori principali come l’accelerometro ed il GPS. Il software ProVAL (Profile Viewing and AnaLysis) è stato utilizzato per calcolare il risultato finale dell’IRI, mentre sono stati sviluppati degli scirpt MATLAB (Matrix Laboratory) per il processamento degli altri indici e per il calcolo iniziale dell’IRI. I risultati ottenuti vengono visualizzati su una mappa interattiva, tramite l’utilizzo delle API Mapbox, in cui, oltre a visualizzare i risultati per ciascun indice, è possibile ottenere delle informazioni da ciascun punto (come il valore associato) cliccando su di esso.  Per quanto riguarda gli indici, i risultati ottenuti mostrano che l’IRI ha una buona relazione tra il valore che è stato calcolato per un determinato segmento stradale e la sua scala di riferimento (IRI scale), nonostante non sia stato possibile identificare un’equazione di correlazione, ma è stata solo simulata. Invece, per quanto riguarda l’indice dei punti critici, esso indentifica in modo accurato i punti maggiormente danneggiati della superfice, che ricadono all’interno di determinate soglie. Infine, gli sviluppi futuri del sistema verranno proposti e discussi.
\clearpage


\end{document}