\documentclass[tesi]{subfiles}
\begin{document}
\begin{flushright}
\section*{Abstract}\label{Abstract_IT}
\vspace{10mm}
\end{flushright}
\noindent In questo lavoro, verr\'a esaminato come migliorare la sicurezza di viaggio su strada, attraverso la collezione e distribuzione di informazioni della superfice stradale utilizzando sensoristica a basso costo. \\ Informazioni sulle condizioni della superfice stradale sono utili a tutti gli utenti della strada ed alle pubbliche amministrazioni per la manutenzione della stessa. 
Il problema considerato che \'e stato considerato \'e quello di individuare le anomalie della superfice stradale, che quando non segnalate possono causare danni all\'\ autovettura, provocano una riduzione del comfort di guida, una minore controllabilit\'a dell\'\ autovettura, oppure possono essere causa di incidenti.\\ In questo lavoro di tesi \'e stato sviluppato un sistema per misurare alcuni indici come per esempio \'l Indice di Rugosit\'a Internazionale (IRI), che rappresenta uno standard per il monitoraggio delle condizioni della superfice stradale, ed altri indicatori come per esempio la localizzazione di punti stradali molto danneggiati.\\La descrizione delle condizioni della superfice stradale \'e effettuata tramite l\'\ analisi di un profilo longitudinale. \\ Per ottenere un profilo di questo tipo \'e stato utilizzato per lo scopo uno smartphone, che \'e stato posizionato all\'\ interno dell\'\ autovettura, e da cui sono stati collezionati e processati alcuni dati derivanti da alcuni dei principali sensori come l\'\ accelerometro ed il GPS. \\ Il software ProVAL \'e stato utilizzato per calcolare il risultato finale dell\'IRI, mentre tramite degli script MATLAB sono stati calcolati e processati gli altri indici, ed anche parte del calcolo dell\'\ IRI. \\ I risultati ottenuti sono stati mostrati all\'\ interno di un sito web, su una mappa interattiva tramite l\'\ utilizzo del API Mapbox, \'e possibile selezionare uno dei differenti indici, ed ottenere delle informazioni (come per esempio il valore associato) da ciascun punto rappresentato sulla mappa cliccando su di esso. \\ In merito agli indici, i risultati ottenuti per l\'\ IRI mostrano che vi \'e una buona relazione tra un determinato segmento di strada e la scala dei valori IRI, anche se tuttavia non \'e stato possibile creare un\'equazione di correlazione, ma \'e stata solo simulata, invece per quanto riguarda i punti critici identificano e localizzando in modo abbastanza accurato i punti maggiormente danneggiati sulla superfice stradale che ricadano all\'\ interno di determinate soglie di calcolo.\\
Future integrazioni del sistema saranno discusse.
\clearpage


\end{document}