\documentclass[tesi]{subfiles}
\begin{document}

\begin{flushright}
\section*{Abstract}\label{Abstract}

\vspace{10mm}
\end{flushright}
\noindent In this work, we examine how to improve traffic safety through collecting and distributing road surface condition information using cheap sensors, in order to provide useful information for road travelers and public institutions for road network maintenance. A system to measure indexes like the International Roughness Index (IRI), and the location of critical points on the road surface, has been developed. The description regarding the quality of the road surface is carried out through the analysis of a longitudinal profile. In particular, a smartphone was used for the purpose, located inside the car, and from which data produced by main sensors, such as the accelerometer and the GPS, are both collected and processed. ProVAL (Profile Viewing and AnaLysis) software was used to calculate the final result of the IRI, while MATLAB (Matrix Laboratory) modules were processed to calculate other indexes and a first part of IRI evaluation. The obtained results are displayed on an interactive map using the Mapbox APIs, it is possible view and select one of the indexes, and get informations (such as the associated value) for each point displayed on the map. Regarding the indexes, the obtained results for the IRI shown that there is a good relationship between the values associated with a given road segment respect IRI scale, even though it was not possible to create a correlation equation but it was just simulated, while the critical point identifies the most damaged points on the road pavements that are within specific thresholds. Eventually, future integration of the system will be proposed and discussed.
\clearpage

\end{document}